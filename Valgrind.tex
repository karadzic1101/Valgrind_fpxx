% Format teze zasnovan je na paketu memoir
% http://tug.ctan.org/macros/latex/contrib/memoir/memman.pdf ili
% http://texdoc.net/texmf-dist/doc/latex/memoir/memman.pdf
% 
% Prilikom zadavanja klase memoir, navedenim opcijama se podešava 
% veličina slova (12pt) i jednostrano štampanje (oneside).
% Ove parametre možete menjati samo ako pravite nezvanične verzije
% mastera za privatnu upotrebu (na primer, u b5 varijanti ima smisla 
% smanjiti 
\documentclass[12pt,oneside]{memoir}

% Paket koji definiše sve specifičnosti mastera Matematičkog fakulteta
\usepackage{matfmaster}
%
% Podrazumevano pismo je ćirilica.
%   Ako koristite pdflatex, a ne xetex, sav latinički tekst na srpskom jeziku
%   treba biti okružen sa \lat{...} ili \begin{latinica}...\end{latinica}.
%
% Opicija [latinica]:
%   ako želite da pišete latiniciom, dodajte opciju "latinica" tj.
%   prethodni paket uključite pomoću: \usepackage[latinica]{matfmaster}.
%   Ako koristite pdflatex, a ne xetex, sav ćirilički tekst treba biti
%   okružen sa \cir{...} ili \begin{cirilica}...\end{cirilica}.
%
% Opcija [biblatex]:
%   ako želite da koristite reference na više jezika i umesto paketa
%   bibtex da koristite BibLaTeX/Biber, dodajte opciju "biblatex" tj.
%   prethodni paket uključite pomoću: \usepackage[biblatex]{matfmaster}
%
% Opcija [b5paper]:
%   ako želite da napravite verziju teze u manjem (b5) formatu, navedite
%   opciju "b5paper", tj. prethodni paket uključite pomoću: 
%   \usepackage[b5paper]{matfmaster}. Tada ima smisla razmisliti o promeni
%   veličine slova (izmenom opcije 12pt na 11pt u \documentclass{memoir}).
%
% Naravno, opcije je moguće kombinovati.
% Npr. \usepackage[b5paper,biblatex]{matfmaster}

% Pomoćni paket koji generiše nasumičan tekst u kojem se javljaju sva slova
% azbuke (nema potrebe koristiti ovo u pravim disertacijama)
\usepackage{pangrami}

% Datoteka sa literaturom u BibTex tj. BibLaTeX/Biber formatu
\bib{matfmaster-primer}

% Ime kandidata na srpskom jeziku (u odabranom pismu)
\autor{Aлександра Караџић}
% Naslov teze na srpskom jeziku (u odabranom pismu)
\naslov{Алат \textit{Valgrind} - имплементација конвенције \textit{FPXX} за архитектуру \textit{MIPS} }
% Godina u kojoj je teza predana komisiji
\godina{2017}
% Ime i afilijacija mentora (u odabranom pismu)
\mentor{др Милена \textsc{Вујошецић Јаничић}, доцент\\ Универзитет у Београду, Математички факултет}
% Ime i afilijacija prvog člana komisije (u odabranom pismu)
\komisijaA{др Филип \textsc{Марић}, ванредни професор\\ Универзитет у Београду, Математички факултет}
% Ime i afilijacija drugog člana komisije (u odabranom pismu)
\komisijaB{др Јелена \textsc{Граовац}, доцент\\ Универзитет у Београду, Математички факултет}
% Ime i afilijacija trećeg člana komisije (opciono)
% \komisijaC{}
% Ime i afilijacija četvrtog člana komisije (opciono)
% \komisijaD{}
% Datum odbrane (obrisati ili iskomentarisati narednu liniju ako datum odbrane nije poznat)
\datumodbrane{15. јануар 2016.}

% Apstrakt na srpskom jeziku (u odabranom pismu)
\apstr{%
%\pangrami
}

% Ključne reči na srpskom jeziku (u odabranom pismu)
\kljucnereci{анализа, геометрија, алгебра, логика, рачунарство, астрономија}

\begin{document}
% ==============================================================================
% Uvodni deo teze
\frontmatter
% ==============================================================================
% Naslovna strana
\naslovna
% Strana sa podacima o mentoru i članovima komisije
\komisija
% Strana sa posvetom (u odabranom pismu)
\posveta{Деди}
% Strana sa podacima o disertaciji na srpskom jeziku
\apstrakt
% Sadržaj teze
\tableofcontents*

% ==============================================================================
% Glavni deo teze
\mainmatter
% ==============================================================================

% ------------------------------------------------------------------------------
\chapter{Увод}
% ------------------------------------------------------------------------------
%\pangrami

%\section{Примери коришћења класичних \LaTeX{} елемената}

% primer korišćenja tabele
%\begin{table}
%\centering
%\caption{Резултати}
%\label{tbl:rezultati}
%\begin{tabular}{c>{\centering}p{2cm}c}
%\toprule
%1 & 2 & 3\\\midrule
%4 & 5 & 6\\\cmidrule(rl){1-2}
%7 & 8 & 8\\
%\bottomrule
%\end{tabular}
%\end{table}

% primer korišćenja slike
%\begin{figure}[!ht]
%  \centering
%  \includegraphics[width=0.5\textwidth]{graph.png}
%  \caption{Графикон}
%  \label{fig:grafikon}
%\end{figure}

% primer jednostavnije matematičke formule
%Ево и један пример математичке формуле: $e^{i\pi} + 1 = 0$. 
% primer referisanja na sliku
%На слици \ref{fig:grafikon} приказан је један графикон.

% primer kompleksnije matematičke formule
%$$
%\int_a^b f(x)\ \mathrm{d}x \ =_{def}\ \lim_{\max{\Delta x_k \rightarrow 0}} \sum_{k=1}^n f(x_k^*)\Delta x_k
%$$

% primer referisanja na poglavlja i strane poglavlja
%Више детаља биће дато у глави \ref{chp:razrada} на страни \pageref{chp:razrada}.

% primer liste
%Можемо правити и набрајања:
%\begin{enumerate}
%\item Анализа 1
%\item Линеарна алгебра
%\item Аналитичка геометрија
%\item Основи програмирања
%\end{enumerate}

%\pangrami

% ------------------------------------------------------------------------------
\chapter{Архитектура \textit{MIPS}}
\label{chp:mips}
% ------------------------------------------------------------------------------
\subsection{MIPS}
MIPS је најелегантнија архитектура међу свим активним RISC архитектура, чак и по мишљењу конкуренције. Елеганција није довољна да би се освојило тржиште, али MIPS микропорцесори су успели бити међу најефикаснији сваком генерацијом остајући међу најједноставнијима.
MIPS процесори је један од RISC процесора, рођеног у плодном периоду академских истраживања и развоја. RISC  (енг. \textit{Reduced Instruction Set Computing}) је атрактивни акроним, 
~\cite{SeeMIPSRun}

\section{Регистри у MIPS-у}

\indent Регистри представљају малу, веома брзу меморију, која је део процесора. MIPS процесори могу вршити операције само над садржајима регистара и специјалним константама које су део инструкције. \\
\indent У MIPS архитектури, постоји 32 регистара опште намене. Само два регистара се понашају другачије од осталих регистара:
\begin{itemize}
  \item \textbf{\$0} - Увек враћа нулу, без обзира коју му се вредност додели
  \item \textbf{\$31} - Увек се користи за адресу повратка из функције на коју се скочи инструкцијом \textit{jal}
\end{itemize}
Сви ови регистри су идентични и могу се користити за било коју истрикцију (може се чак користити и регистар \$0 као дестинација, мада ће резултат да нестане). 

Регистри опште намене су описани у наставку:
\begin{itemize}
  \item \textbf{at} - Резервисан за псеудоинструкције које асемблер генерише
  \item \textbf{v0, v1} - Користи се за враћање резултата при повратку из неке функције. Резултат може бити целобројног типа или број записан у фиксном зарезу.
  \item \textbf{а0 - а3} - Користи се за прослеђивање прва 4 аргумената функције која се позива
  \item \textbf{t0 - t9} - по конвенцији која је описана 
  \item \textbf{s0 - s7} - по конвенцији која је описана 
  \item \textbf{k0, k1} - Резервисано за систем прекида, који након коришћења не враћа садржај ових регистара на почетни. Како се прекид не позива из програма који се тренутно извршава, нема примене позивне конвенције. То значи да се садржај регистара које прекинути програм користи може пореметити. Због тога, систем прекида прво сачува саджаје регистара опште намене, који су важни за програм који се у том тренутку извршавао, и чији садржај планира да мења. У те сврхе се користе ови регистри.
  \item \textbf{gp} - Користи се у различите сврхе. У коду који  не зависи од позиције (енг. \textit{Position Independent Code} скраћено PIC), свом коду и подацима се пристпуа преко табеле показивача, познате као GOT (скраћено од енг. \textit{Global Offset Table}). Регистар \textit{\$gp} показује на ту табелу. PIC је код који се може извршавати на било којој меморијској адреси, без модификација. PIC се најчешће користи за дељење библиотеке, при чему се заједнички код библиотеке може учитати у одговарајуће локације адресних простора различитих програма који је користе. \\
  У регуларном коду који зависи од позиције, регистар \textbf{\$gp} се користи као показивач на средину у статичкој меморији. То значи да се подацима који се налазе 32 КВ лево или десно од адресе која се налази у овом регистру може приступити помоћу једне инструкције. Дакле, инструкције \textit{load} и \textit{store} које се користе за учитавање, односно складиштење података, се могу извршити у само једној инструкцији, а не у две као што је иначе случај. У пракси се на ове локације смештају глобали подаци који не заузимају много меморије. Оно што је битно је да овај регистар не користи сви системи за компилацију и сва окружења за извршавање.
  
  \item \textbf{sp} - Показивач на стек. Оно што је битно је да стек расте наниже. Потребне су специјалне инструкције да би се показивач на стек повећао и смањио, тако да \textit{MIPS} ажурира стек само при позиву и повратку из фукције, при чему је за то одговорна функција која је позвана. \textit{sp} се при уласку у функцију прилагођава на најнижу тачку на стеку којој ће да приступати у функцији. Тако ј еомогућено да компилатор може да приступи поменљивама на стеку помоћу константног помераја у односу на \textit{\$sp}.
  \item \textbf{fp} - Познат и као \textit{\$s8}, показивач на стек оквир. Користи се од стране функције, за праћење стања на стеку, за случај да из неког разлога компилатор или програмер не могу да израчунају померај у односу на \textit{\$sp}. То се може догодити уколико програм врши проширење стека, при чему се вредност проширења рачуна у току извршавања. Ако се дно стека не може израчунати у току превођења, не може се приступити променљивама помоћу \textit{\$sp}, па се на почетку функције \textit{\$fp} иницијализује на константну позицију која одговара стек оквиру функције. Ово је локално за функцију.
  \item \textbf{ra} - Ово је подразумевани регистар за смештање адресе воратка и то је подржано кроз одговарајуће инструкције скока. Ово се разлицкује од конвеција које се корсите на архитеткурама џ86, где инструкција позива функције адресз повратка смешта на стек. При уласку у фукцију регистар \textit{ra} обично садржи адресу повратка фукције, тако да се функције углавном завршавају инструкцијом \textit{jr \$ra}, али у принципу, може се користити и неки други регистар. Због неких оптимизација које врши процесор, препоручује се коришћење регистара \textit{\$ra}. Функције које позивају друге функције морају сачувати садржај регистара \textit{\$ra}.
\end{itemize}

\indent Постоје два специјална регистра \textit{Hi} и \textit{Lo}, који се користе само при множењу и дељењу. ово нису регистри опште намене, те се не користе при другим инструцијама. Не може им се приступити директно, већ постоје специјалне инструкције \textit{mfhi} и \textit{mflo} за премештање садршаја ових регистара. Инструкција \textit{mfhi} је облика \textit{mfhi rd}, и она премешта садржај регистар \textit{Hi} у регистар \textit{rd}, док инструкција \textit{mflo} премешта садржај регистара \textit{Lo}.


\section{Floating point регистри у MIPS-у}

MIPS архитектура користи два формата FP (скр. \textit{Floating Point}) препоручена од стране IEEE 754:

\begin{itemize}
	\item \textit{Једнострука прецизност} (eнг. \textit{Single precision}) - Користи се 32 бита за чување у меморији. MIPS компајлери користе једноструку прецизност за променљиве типа \textit{float}
	\item \textit{Двострука прецизност} (eнг. \textit{Double precision}) - Користи се 64 бита за чување у меморији. C компајлери коисте двоструку прецизност за C \textit{double} типове података.
\end{itemize}

\indent Начин да се две речи ширине 32 бита се смештају у меморију као једна реч ширине од 64 бита је начин смештања у меморији (виша половина битова прво, или нижа половина битова прво) и зависи од начина смештања у меморији. 

\indent Стандартд IEEE 754 је веома захтеван и поставио је два велика проблема. Први, омогућавање детекције неуобичајних резултата доводи проточну обраду (енг. \textit{pipeline}) тешком. Постоји опција да се имплементира IEEE механизам сигнализирања изузетака, али је проблем да се детектују случајеви када хардвер не може да произведе исправан резултат и потребна му је помоћ.

\indent Када се IEEE изузетак деси требало би обевестити и корисника, ово би требало бити синхроно; након заустављања корисник би желио да види све предходно извршене инструкције и све FP регистре који су у preinstruction стању и желе да се увере да ни једна следећа инструкција нема никакав ефекат.

\indent У \textit{MIPS} архитектури, хардверска заустављања су била овако одрађена. Ово заправо ограничава могућности проточне обраде FP операција, јер се не може извршити сљедећа инструкције све док хардвер може бити сигуран да операција FP неће произвести заустављање. Зарад избегавања додавања времена за извршавање, FP операције морају да одлуче да ли ће доћи до заустављања у првој фази. 

\section{Системски позиви}




\section{FPXX конвеција}

% ------------------------------------------------------------------------------
\chapter{Valgrind}
\label{chp:valgrind}

\indent \textit{Valgrind} је платформа за прављење алата за динамичку бинарну анализу кода. Динамичка анализа обухвата анализу корисничког програма у извршавању, док бинаран анализа обухвата анализу на нивоу машинског кода, снимљеног или као објектни код (неповезан) или као извршни код (повезан). Постоје \textit{Valgrind} алати који могу аутоматски да детектују проблеме са меморијом, процесима као и да изврше оптимизацију самог кода. \textit{Valgrind} се може користити и као алат за прављење нових алата. \textit{Valgrind} дистрибуција тренутно броји следеће алате: детектор меморијских грешака, детектор грешака нити, оптимизатор скривене меморије и скокова, генератор графа скривене меморије и предикције скока и оптимизатор коришћења динамичке меморије. \textit{Valgrind} ради на следећим архитектурама: \textbf{\textit{X86/Linux, AMD64/Linux, ARM/Linux, ARM64/Linux, PPC32/Linux, PPC64/Linux, PPC64LE/Linux, S390X/Linux, MIPS32/Linux, MIPS64/Linux, X86/Solaris, AMD64/Solaris, ARM/Android (2.3.x и новије), ARM64/Android, X86/Android (4.0 и новије), MIPS32/Android, X86/Darwin and AMD64/Darwin (Mac OS X 10.12)}}. \\

\indent У наредним поглављима биће детаљно описана структура \textit{Valgrind} и његових алата, као и начин употребе са примерима проблема са којима се програмери свакодневно сусрећу.

\section{O Valgrindu}

\indent Алат за динамичку анализу кода се креира као додатак, писан у C програмског језику, на језгро \textit{Valgrind}. 


\begin{center}
\textit{Језгро Valgrinda + алат који се додаје = Алат Valgrinda} 
\end{center}


\indent Језгро \textit{Valgrind}-а омогућава извршавање клијетског програма, као и снимање извештаја који су настали приликом анализе самог програма. 

\indent Алати \textit{Valgrind}-а користе методу бојења вредности. Они заправо сваки регистар и меморијску вредност "боје" (замењују) са вредношћу која говори нешто додатно о оригиналној вредности. 

\indent Сви \textit{Valgrind} алати раде на истој основи, иако информације које се емитују варирају. Информације које се емитују могу се искористити за отклањање грешака, оптимизацију кода или било коју другу сврху за коју је алат дизајниран.

\indent Сваки \textit{Valgrind}-ов алат је статички повезана извршна датотека која садржи код алата и код језгра. Извршна датаоке valgrind представља програм омотач који је на основу --tool опције бира алат који треба покренути и покреће га помоћу системског позива \textbf{\textit{execve}}. Извршна датотека алата статички је линкована тако да се учитава почев од неке адресе која је обично доста изнад адресног простора који користе класичан кориснички програм (на \textbf{\textit{x86/Linux}} и \textbf{\textit{MIPS/Linux}} користи се адреса 0x38000000). У ретким случајевима, када та адреса није потреба, \textit{Valgrind} се може прекомпајлирати да користи неку другу адресу. \textit{Valgrind}-ово језгро прво иницијализује под-систем као што су менаџер адресног простора, и његов унутрашњи алокатр меморије и затим учитава клијентову извршну датотеку. Потом се иницијализују \textit{Valgrind}-ови субсистеми као што су транслациона табела, апарат за обраду сигнала, распоређивач нити и учитавају се информације за дебаговање клијента, уколико постоје. Од тог тренутка \textit{Valgrind} има потпуну контролу и почиње са превођењем и извршавањем клијентског програма. Може се рећи да \textit{Valgrind} врши JIT (\textit{Just In Time}) превођење машинског кода програма у машинкси код програма допуњен инструментацијом. Ниједан део кода клијента се не извршава у свом изворном облику. Алат се умеће у оригинални код на почетку, затим се нови код преводи, сваки основни блок појединачно, који се касније извршава. Процес превођења се састоји из рашчлањивања оригиналног машинског кода у IR (скр. \textit{intermediate representation}) који се касније инструментализује са алтом и поново преводи у нови машински код. 

\indent Резултат свега овога се назива транслација, која се чува у меморији и која се извршава по потреби. Језгро троши највише времан на сам процес прављења, проналажења и извршавања транслације. Оригинални код се никада се извршава. Једини проблем који се овде може догодити је ако се врши транслација кода који се мења током извршавања програма.

\indent IR има неке \textit{RISC} одлике као што су \textit{load/store}, свака операција ради само једну ствар, кад се линеаризује све операције раде само на привременим промељивама и литералима. Да би се подржале све целобројне, FP и SIMD операције над различитим величинама потребно је више од 200 примитвних аритметичко-логичких инструкција. 

\indent Постоје многе компликације које настају приликом смештања два програма у један процс (клијентски програм и програм алата). Многи ресурси се деле између ова два програма, као што су регистри или меморија. Такође, алат \textit{Valgrind}-а не сме да се одрекне тоталне контроле над извршавањем клијетског програма приликом извршавања системских позива, сигнала и нити.

\subsection{Основни блок}

\indent \textit{Valgrind} дели оригинални код у секвенце које се називају основни блокови. Основни блок је праволинијска секвенца машинског кода, на чији се почетак скаче, а која се завршава са скоком, позивом функције или повратком. Сваки код програма који се анализира поново се преводи на захтев, појединачно по основним блоковима, непосредно пре самог извршавања самог блока. Ако узмемо да су основни блокови клијетског кода \textit{BB1, BB2, ...} онда преведене основне блокове обележавамо са \textit{t(BB1), t(BB2), ...} Величина основног блока је ограничена на максимално 60 машинских инструкција. На процесорима \textit{MIPS}, инструкције скока и гранања имају такозвано "одложено извршавање". То значи да се приликом извршавања тих инструкција извршава и инструкција која се налази непосредно иза инструкције гранања или скока. У случају да је последња шесдесета инструкција основног блока инструкција гранања, \textit{Valgrind} учитава и инструкцију која се налази непосредно иза ње, односно шесдесет и прва инструкција. Тиме се омогућава конзистентно извршавање програма који се анализира, као и у случају да се програм извршава без посредства \textit{Valgrind}-а. Уколико након извршених 60 инструкција \textit{Valgrind} није наишао на инструкцију гранања, секвенца инструкција се дели на два или више основних блокова, који се извршавају један за другим.



\subsection{Системски позиви}

\indent Апликациони програми комуницирају са оперативним системом помоћу системских позива (eнг. \textbf{system calls}), тј. преко операција (функција) које дефинише оперативни систем.
\indent Системски позиви се реализују помоћу система прекида: кориснички програм поставља параметре системског позива на одређене меморијске локације или регистре процесора, иницира прекид, оперативни систем преузима контролу, узима парамтре, извршава тражене радње, резултат ставља на одређене меморијске локациј еили у регистре и враћа контролу корисчком програму. 
\indent Апликација која жели да користи неке од ресурса, као што су меморија, процесор или улазно/излазни уређаји, комуницира са језгром опративног система користећи системске позиве. Језгро оператвиног система дели виртуелну маморију на корисничку меморију и системску меморију. Системска меморија је одређена за само језгро оператвиног система, његова проширања, као и за урављачке програме. Кориснички прогстор је део меморије где се налазе све корсничке апликације приликом њиховог изврђавања. Корисничке апликације могу да приступе улазно/излазним уређајима, виртуелној меморији, датотекама ид ругим реуссрисам језгра оператвино система користећи само системске позиве. Системски позиви обезбеђују спрегу између програма који се извршава и оператвиног система. Генерално, реализују се на асемблерском језику, али новији виши програсмки језици, попут језика C и C++, такође омогућавају реализацију системског позива. Пграом кои се извршава може проседити параметре опративном систему на три начина:
\begin{itemize}
  \item прослеђивање параметара у регистрима процесора;
  \item постављањем параметара у меморијској табели, при чему се адреса табеле прослеђује у регистру процесора;
  \item постављањем параметара на врх стека (енг. \textit{push}), које оператвни систем "скида" (енг. \textit{pop}).
\end{itemize}
\indent Системски позиви се извршавају без посредства \textit{Valgrind}-а, зато што језгро \textit{Valgrind}-а не може да прати њихово извршавање у самом језгру оперативног система.


\subsection{Транслација}

\indent У наставку су описани кораци које \textit{Valgrind} извршава приликом анализе програма. Постоји осам фаза транслације. Све фазе осим инструментацје коју обавља алат \textit{Valgrind}-а, обавља језгро \textit{Valgrind}-а.

\begin{itemize}
  \item \textbf{ Дисасемблирање } - процес превођења машинског кода у еквивалетни асемблерски код. \textit{Valgrind} врши превођење машинског кода у интерни скуп инструкција која се називају међукод инструкције. Међукод представља редуковани скуп инструкција (скр. енг. \textit{RISC}). Ова фаза је зависна од архитетктуре на којој се извршава.
  \item \textbf{ Оптимизација 1} - Прва фаза оптимизације линеаризује \textit{IR} репрезентацију. Примењују се неке стандардне оптимизације програмских преводилаца као што су уклањање редудантног кода, елиминација подизраза, једноставно одмотавање петљи и сл.
  \item \textbf{ Инструментација} - Блок кода у \textit{IR} репрезентацији се прослеђује алату, који може произвољно да га трансформише. Приликом инструментације алат у задати блок додаје додатне \textit{IR} операције, кјима проверава исправност рада програма.
  \item \textbf{ Оптимизација 2 } - Друга фаза оптимизације је једноставније од прве, укључује множење констати и уклањање мртвог кода.
  \item \textbf{ Градња стабла } - Линеаризована \textit{IR} репрезентација се конвертује натраг у стабло ради лакшег избора инструкција.
  \item \textbf{ Одабир инструкција } - Стабло \textit{IR} репрезентације се конвертује у листу инструкција које користе виртуалне регистре. Ова фаза се такође разликује у зависности од архитеткуре на којој се извршава. 
  \item \textbf{ Алокација регистара} - Виртуални регистри се замењују стварним. По потреби се уводе пребацивања у меморију. Независна је за платформу, користи позив функција које налазе из који се регистара врши читање и у које се врши упис.
  \item \textbf{ Асемблирање } - Изабране инструкције се енкодују на одговарајући начин и смештају у блок мемроји. Ова фаза се такође разликује у зависноти од архитектуре на који се изршава. ~\cite{SeeMIPSRun}
\end{itemize}

\section{Memcheck}

\section{Callgrind}

\section{Cachgrind}

\section{Massif}

\section{Helgrind}

\section{DRD}

\chapter{FPXX}
\label{chp:fpxx}

\chapter{Закључак}

% ------------------------------------------------------------------------------
\pangrami

\pangrami

% ------------------------------------------------------------------------------
% Literatura
% ------------------------------------------------------------------------------
\literatura

% ==============================================================================
% Završni deo teze i prilozi
\backmatter
% ==============================================================================

% ------------------------------------------------------------------------------
% Biografija kandidata
\begin{biografija}
\textbf{Вук Стефановић Караџић} (\emph{Тршић, 26. октобар/6. новембар
  1787. — Беч, 7. фебруар 1864.}) био је српски филолог, реформатор
српског језика, сакупљач народних умотворина и писац првог речника
српског језика.  Вук је најзначајнија личност српске књижевности прве
половине XIX века. Стекао је и неколико почасних доктората.
Учествовао је у Првом српском устанку као писар и чиновник у
Неготинској крајини, а након слома устанка преселио се у Беч,
1813. године. Ту је упознао Јернеја Копитара, цензора словенских
књига, на чији је подстицај кренуо у прикупљање српских народних
песама, реформу ћирилице и борбу за увођење народног језика у српску
књижевност. Вуковим реформама у српски језик је уведен фонетски
правопис, а српски језик је потиснуо славеносрпски језик који је у то
време био језик образованих људи. Тако се као најважније године Вукове
реформе истичу 1818., 1836., 1839., 1847. и 1852.
\end{biografija}
% ------------------------------------------------------------------------------

\end{document} 